\section{Korrosion}
    Metall reagiert als RM: \ce{Me <--> $Me^{z+}$ + z e-}
    
    Möglich wenn $\Delta$G $<$ 0 $\&$ v.a. \ce{O2, H2O/H3O+} (OM) vorhanden

\subsection{Korrosionsarten}
    \subsubsection{Elchem Korrosion}
        $\Rightarrow$ Bildung galvanische Zelle
    \subsubsection{O2-Typ-Korrosion}
        \ce{O2 + 2H2O + 4e- -> 4OH-}\\
        \ce{2Fe + 3/2 O2 + H2O -> 2FeOOH}\\
        Voraussetzung ist Vorhandensein von \ce{O2} und \ce{H2O}; RG relativ langsam!
    \subsubsection{Säure/Wasserstoffkorrosion}
        Ist pH-Abhängig:
        \begin{itemize}
            \item Sauer: \ce{2H3O+ + 2 e- <--> H2 + 2H20}
            \item Basisch: \ce{2H2O + 2 e- <--> H2 + 2 OH-}
        \end{itemize}
    \subsubsection{Beispiele Al H-Typ-Korrosion}
        \textbf{H2 Korrosion von Al in basischer Lösung}\\
        \begin{tabular}{p{1.8cm}ccc}
            Depassivierung: & \ce{Al2+2OH-+3H2O}        & \ce{->} & \ce{2[AL(OH)4]-(aq)}\\
            Oxidation:      & \ce{Al}                   & \ce{->} & \ce{Al3+ + 3e-}\\
            Reduktion:      & \ce{2H2O + 2 e-}          & \ce{->} & \ce{H2 + 2OH-}\\
            Redoxreaktion:  & \ce{2 Al + 6 H2O + 2OH-}  & \ce{->} & \ce{2[Al(OH)4]- + 3H2}
        \end{tabular}

        \textbf{H2 Korrosion von Al in saurer Lösung}\\
        \begin{tabular}{p{1.8cm}ccc}
            Oxidation:      & \ce{Al}               & \ce{->} & \ce{Al3+ + 3e-}\\
            Reduktion:      & \ce{2H3O+ + 2e-}      & \ce{->} & \ce{H2 + H2 + 2 H2O}\\
            Redoxreaktion:  & \ce{2 Al + 6 H3O+}    & \ce{->} & \ce{2 Al3+ + 6H2O + 3H2}
        \end{tabular}
\subsection{Oxidschichten}
    Metallische Werkstoffe (ausser Gold/Platinmetalle) bilden bei Raumtemperatur mit Luft eine Oxidschicht, es entsteht ein Metalloxid:\\
    \ce{n Me + \frac{m}{2} O2 -> $Me_nO_m$}\\
    Der Schutzfaktor kann mittels PBV ermittelt werden:\\
    $PBV = \frac{V(Metalloxid)}{V(Metall)}$\\
    \begin{itemize}
        \item PBV $\ll$ 1: Rissige, nicht schützende Schicht (MG(0.8), Na (0.3))
        \item PBV 1- 2: Kompakte, schützende Oxidschicht (Al(1.3), Ni(1.5), Ti(1.7), Cu(1.7), Cr(2.1), Fe(2.1))
        \item PBV $\gg$ 2: Abblätternde nicht schützende Schicht (V(3.2), W(3.4), Rost(3.6))
    \end{itemize}
\subsection{Ablauf der Korrosion in wässrigen Lösungen}
    Alle Korrosionsreaktionen verlaufen in 2 Teilschritten:
    \begin{itemize}
        \item Depassivierung
        \item Eigentliche Korrosion
    \end{itemize}

    Voraussetzungen für Korrosion:
    \begin{itemize}
        \item Metall ist in Elektrolytlösung eingetaucht
        \item Metall ist von dünnem Flüssigkeitsfilm bedekt.\\
        Können durch Regen, Tau, Bodenfeuchtigkeit oder rel. Luftfeuchtigkeit $> 70\%$ entstehen. Bei Oberflächen mit hygroskopischen Salzen kann auch früher Korrosion entstehen.
    \end{itemize}
\subsection{Passivatoren und Depassivatoren}
    Depassivierung hängt vom Gehalt von Passivatoren und Depassivatoren in Elektrolytlösung ab
    \subsubsection{Passivatoren}
        $\Rightarrow$ bieten \textbf{anodischen Schutz}($E_A$ wird vergrössert)
        \begin{itemize}
            \item Fe: \ce{OH-, CrO4^{2-}, NO2-}
            \item Al: \ce{NO3-}
        \end{itemize}
        
    \subsubsection{Depassivatoren}
        $\Rightarrow$ \textbf{zerstören Passivoxidfilm}, bewirken (oft lokale \textbf{Depassivierung}($E_A$ wird verkleinert))
        \begin{itemize}
            \item Fe: \color{blue} \ce{Cl-}\color{black}, \color{red} \ce{H3O+}\color{black}, \ce{SO4^2-}
            \item Al: \color{blue} \ce{Cl-}\color{black}, \color{red} \ce{H3O+}\color{black}, \ce{OH-}
            \item Cu: \color{blue} \ce{Cl-}\color{black}, \color{red} \ce{H3O+}\color{black}, \ce{NH3}
            \item Ni: \color{blue} \ce{Cl-}\color{black}, \color{red} \ce{H3O+}\color{black}
        \end{itemize}
        
\subsection{Potentialverhältnisse/Aktivierungsenergie}
    Wann korrodieren Metalle nach \ce{H2}/\ce{O2}-Typ?\\
    $\Rightarrow \Delta$G $<$ 0\\
    $\Rightarrow$ Korrosion abhängig von E(M/\ce{M^z+}) unsd E(OM)\\
    E(OM) ist pH-abhängig:\\
    $E_{H2}$ = -0.059*pH\\
    $E_{O2}$ = 1.23 - 0.059*pH\\
\subsection{Kontaktkorrosion}
    \begin{itemize}
        \item Reduktion von \ce{O2} an gesamter Oberfläche
        \item Oxidation nur an unedlerem Metall $\rightarrow$ verstärkte Korrosion
        \item Edleres Metall $\rightarrow$ keine Korrosion(kathodisch geschützt)
        \item Flächenregel: $\frac{v_k(Zn)}{v_k(Zn + Fe)} = \frac{A(Zn)}{A(Zn + Fe)}$
    \end{itemize}

\subsection{Lochfrasskorrosion}
    \begin{itemize}
        \item Stark lokalisierte Korrosion
        \item Bildung enger tiefer Löcher
        \item schwer erkennbar
    \end{itemize}

\subsection{Belüftungselemente}
    \begin{itemize}
        \item Kann nur bei passivierbaren Metallen auftreten!
        \item Für Passivschicht ist \ce{O2} notwendig
        \item An engen Stellen kann \ce{O2}-Zufuhr erschwert werden $\rightarrow$ Depassivierung $\Rightarrow$ Lochfrass
        \item Zusätzlich Flächenregel (Spalt $\rightarrow$ kleine Anode, Passivoxidschicht $\rightarrow$ grosse Kathode)
    \end{itemize}

