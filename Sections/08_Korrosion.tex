\section{Korrosion}
Metall reagiert als RM: \ce{Me <--> $Me^{z+}$ + z e-}, Möglich wenn $\Delta$G $<$ 0 $\&$ v.a. \ce{O2, H2O/H3O+} (OM) vorhanden.
\subsection{Korrosionsarten}
\subsubsection{Elchem Korrosion}
$\Rightarrow$ Bildung galvanische Zelle
\subsubsection{O2-Typ-Korrosion}
\ce{O2 + 2H2O + 4e- -> 4OH-}\\
\ce{2Fe + 3/2 O2 + H2O -> 2FeOOH}\\
Voraussetzung ist vorhandensein von \ce{O2} und \ce{H2O}; RG relativ langsam!
\subsubsection{Säure/Wasserstoffkorrosion}
Ist pH-Abhängig:
- Sauer: \ce{2H3O+ + 2 e- <--> H2 + 2H20}\\
- Basisch: \ce{2H2O + 2 e- <--> H2 + 2 OH-}\\
\subsubsection{Beispiele Al H-Typ-korrosion}
\textbf{H2 Korrosion von Al in basischer Lösung}\\
\begin{tabular}{p{1.8cm}c}
Depassivierung: & \ce{Al2+2OH-+3H2O -> 2[AL(OH)4]-(aq)}\\
Oxidation:      & \ce{Al -> Al3+ + 3e-}\\
Reduktion:      & \ce{2H2O + 2 e- -> H2 + 2OH-}\\
Redoxreaktion:  & \ce{2 Al + 6 H2O + 2OH- -> 2[Al(OH)4]- + 3H2}
\end{tabular}
\textbf{H2 Korrosion von Al in saurer Lösung}\\
\begin{tabular}{p{1.8cm}c}
Oxidation:      & \ce{Al -> Al3+ + 3e-}\\
Reduktion:      & \ce{2H3O+ + 2e- -> H2 + H2 + 2 H2O}\\
Redoxreaktion:  & \ce{2 Al + 6 H3O+ -> 2 Al3+ + 6H2O + 3H2}
\end{tabular}
\subsection{Oxidschichten}
Metallische Werkstoffe (Ausser Gold/Platinmetalle) bilden bei Raumtemperatur mit Luft eine Oxidschicht, es entsteht ein Metalloxid:\\
\ce{n Me + \frac{m}{2} O2 -> $Me_nO_m$}\\
Der schutzfaktor kann mittels PBV ermittelt werden:\\
$PBV = \frac{V(Metalloxid)}{V(Metall)}$\\
\begin{itemize}
    \item PBV $\ll$ 1: Rissige, nicht schützende Schicht\\
    bsp. MG(PBV = 0.8), Na (0.3)
    \item PBV 1- ca.2: Kompakte, schützende Oxidschicht\\
    bsp. Al(1.3), Ni(1.5), Ti(1.7), Cu(1.7), Cr(2.1), Fe(2.1)
    \item PBV $\gg$ 2: Abbläternde nicht schützende Schicht\\
    bsp. V(3.2), W(3.4), Rost(3.6)
\end{itemize}
\subsection{Ablauf der Korrosion in wässrigen Lösungen}
Alle Korrosionsreaktionen verlaufen in 2 Teilschritten:\\
\begin{itemize}
    \item Depassivierung
    \item Eigentliche Korrosion
\end{itemize}
Voraussetzungen für Korrosion:\\
\begin{itemize}
    \item Metall ist in Elektrolytlösung eingetaucht
    \item Metall ist von dünnem Flüssigkeitsfilm bedekt.\\
    Können durch Regen, Tau, Bodenfeuchtigkeit oder rel. Luftfeuchtigkeit $> 70\%$ entstehen. Bei Oberflächen mit hygroskopischen Salzen kann auch früher Korrosion entstehen.
\end{itemize}
\subsection{Passivatoren und Depassivatoren}
Ob Depassivierung eines met. werkstoffes möglich ist ist vom Gehalt von Passivatoren und Depassivatoren in Elektrolytlösung.
\subsubsection{Passivatoren}
$\Rightarrow$ bieten \textbf{anodischen Schutz}($E_A$ wird vergrössert)\\
Passivatoren für Fe: \ce{OH-, CrO4^{2-}, NO2-}\\
Passivator für Al: \ce{NO3-}
\subsubsection{Depasivatoren}
$\Rightarrow$ \textbf{zerstören Passivoxidfilm}, bewirken (oft lokale \textbf{Depassivierung}($E_A$ wird verkleinert)\\
Depassivatoren für Fe: \color{blue} \ce{Cl-}\color{black}, \color{red} \ce{H3O+}\color{black}, \ce{SO4^2-}\\
Depassivatoren für Al: \color{blue} \ce{Cl-}\color{black}, \color{red} \ce{H3O+}\color{black}, \ce{OH-}\\
Depassivatoren für Cu: \color{blue} \ce{Cl-}\color{black}, \color{red} \ce{H3O+}\color{black}, \ce{NH3}\\
Depassivatoren für Ni: \color{blue} \ce{Cl-}\color{black}, \color{red} \ce{H3O+}\color{black}\\
\subsection{Potentialverhältnisse/Aktivierungsenergie}
Wann korrodieren Metalle nach \ce{H2}/\ce{O2}-Typ?\\
$\Rightarrow \Delta$G $<$ 0\\
$\Rightarrow$ Korrosion abhängig von E(M/\ce{M^z+}) unsd E(OM)\\
E(OM) ist pH-abhängig:\\
$E_{H2}$ = -0.059*pH\\
$E_{O2}$ = 1.23 - 0.059*pH\\
\subsection{Kontaktkorrosion}
Reduktion von \ce{O2} an gesamter Oberfläche
Oxidation nur an unedlerem Metall $\rightarrow$ verstärkte Korrosion\\
Edleres Metall $\rightarrow$ keine Korrosion(kathodisch geschützt)\\
Flächenregel: $\frac{v_k(Zn)}{v_k(Zn + Fe)} = \frac{A(Zn)}{A(Zn + Fe)}$
\subsection{Lochfrasskorrosion}
Stark lokalisierte Korrosion\\
Bildung enger tiefer Löcher\\
schwer erkennbar\\
\subsection{Belüftungselemente}
Kann nur bei passivierbaren Metallen auftreten!\\
Für Passivschicht ist \ce{O2} notwendig\\
An engen stellen kann \ce{O2} zufuhr erschwert werden. Dies bewirkt lokale Depassivierung $\rightarrow$ lochfrass\\
Zusätzlich Flächenregel(Spalt = kleine Anode, Passivoxidschicht = grosse Kathode
