\section{Aufbau der Stoffe}

\subsection{Grundlagen}
\begin{center}
    \begin{tabular}{|c|c|c|}\hline
        Atomare Masseeinheit:               & Elementarladung:         & max. $2 \cdot n^{2}$ Elektronen \\ 
        $u = \frac{1}{6.022}10^{-23}g$      & $e = \pm 1.6022 \cdot 10^{-19}C$ & pro Energieniveau $n$ \\ \hline
    \end{tabular}
\end{center}


\begin{minipage}{0.25\linewidth}
    $ \prescript{\text{Atommasse }}{\text{Ordnungszahl }}{X}^{\text{ Ladung}} $
\end{minipage}
\hfill
\begin{minipage}{0.65\linewidth}
    \begin{itemize}[itemsep=1pt, parsep=0pt]
        \item Ordnungszahl = Protonenzahl
        \item Atommasse = Summe Protonen \& Neutronen
        \item Ladung = Summe Protonen \& Elektronen
    \end{itemize}
\end{minipage}

Bausteine: Protonen \& Neutronen im Kern, Elektronen in der Hülle

\subsection{Valenzelektronen (Ve) ...}

\begin{itemize}[itemsep=1pt, parsep=0pt]
    \item -Anzahl anhand der \textbf{Hauptgruppen} im PSE bestimmbar
    \begin{itemize}
        \item Natrium (Na): 1. Hauptgruppe = 1 Ve
        \item Kohlenstoff (C): 4. Hauptgruppe = 4 Ve 
    \end{itemize}
    \item -Bestimmung der \textbf{Nebengruppen} komplizierter/unmöglich \\$\Rightarrow$ verschiedene Formen möglich
    \item haben starken Einfluss auf chemische Eigenschaften
    \item variieren zw. 1 und 8 $\rightarrow$ \textbf{Oktett-Regel}\\$\rightarrow$ Atome wollen äusserste Schale voll haben\\$\rightarrow$ Elektronen abgeben oder aufnehmen $\Rightarrow$ chemische Reaktion
\end{itemize}

% Anzahl Valenzelektronen (Ve) kann anh. der \textbf{Hauptgruppen} im PSE bestimmt werden:
% \begin{itemize}
%     \item Natrium (Na): 1. Hauptgruppe = 1 Ve
%     \item Kohlenstoff (C): 4. Hauptgruppe = 4 Ve
% \end{itemize}
% Für die Elemente der \textbf{Nebengruppe} ist die Bestimmung der Ve-Anzahl \textbf{komplizierter/unmöglich} $\rightarrow$ verschiedene Formen möglich

% Die chemischen Eigenschaften der Elemente sind stark abhängig von der Anzahl-Ve 

% \textbf{Atome streben die Oktett-Regel an!}

% Das bedeutet das die Atome steht die äusserste Schale (Valenz-Schale) voll haben möchten. 
% Um diesen Zustand zu erreichen werden Elektronen aufgenommen oder abgegeben(chemische Reaktion).

\subsection{Lewis-Formel $\rightarrow$ gibt nur Valenzelektronen an}
\begin{center}
    \lewis{Li}{}{}{}{.}{}{1s^22s^1} +
    \lewis{F}{..}{.}{..}{..}{}{1s^22s^22p^5}
    $\longrightarrow$
    \lewis{Li}{}{}{}{}{+}{1s^2}
    \lewis{F}{..}{..}{..}{..}{-}{1s^22s^22p^6}
    $\quad$(= LiF)
\end{center}
% \vspace{-0.6cm}
