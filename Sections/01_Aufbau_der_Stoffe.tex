\section{Aufbau der Stoffe}

\subsection{Grundlagen}
\begin{center}
    \begin{tabular}{|c|c|c|}\hline
        Atomare Masseeinheit:               & Elementarladung:         & max. $2 \cdot n^{2}$ Elektronen \\ 
        $u = \frac{1}{6.022}10^{-23}g$      & $e = \pm 1.6022 \cdot 10^{-19}C$ & pro Energieniveau $n$ \\ \hline
    \end{tabular}
\end{center}


\begin{minipage}{0.25\linewidth}
    $ \prescript{\text{Atommasse }}{\text{Ordnungszahl }}{X}^{\text{ Ladung}} $
\end{minipage}
\hfill
\begin{minipage}{0.65\linewidth}
    \begin{itemize}[itemsep=1pt, parsep=0pt]
        \item Ordnungszahl = Protonenzahl
        \item Atommasse = Summe Protonen \& Neutronen
        \item Ladung = Summe Protonen \& Elektronen
    \end{itemize}
\end{minipage}

Bausteine: Protonen \& Neutronen im Kern, Elektronen in der Hülle

\subsection{Valenzelektronen}
Die Anzahl Valenzelektronen (V.e.) kann anhand der Hauptgruppen aus dem PSE ausgelesen werden:
\begin{itemize}
    \item Natrium(Na); erste Hauptgruppe = 1 V.e.
    \item Kohlenstoff(C); 4. Hauptgruppe = 4 V.e.
\end{itemize}  
Für die Elemente der Nebengruppe wird die bestimmung der V.e.-Anzahl komplizierter/unmöglich da diese in verschiedenen Formen vorkommen können.

Die chemischen Eigenschaften der Elemente sind stark abhängig von der Anzahl-V.e. 

\textbf{Atome streben die Oktett-Regel an!}

Das bedeutet das die Atome steht die äusserste Schale (Valenz-Schale) voll haben möchten. 
Um diesen Zustand zu erreichen werden Elektronen aufgenommen oder abgegeben(chemische Reaktion).

\subsection{Lewis-Formel $\rightarrow$ gibt nur Valenzelektronen an}
    % In dieser Schreibweise werden nur die Valenzelektronen der Elemente angegeben. Dies hilft dabei die entstehenden Bindungen zu visualisieren.
\begin{center}
    \lewis{Li}{}{}{}{.}{}{1s^22s^1} +
    \lewis{F}{..}{.}{..}{..}{}{1s^22s^22p^5}
    $\longrightarrow$
    \lewis{Li}{}{}{}{}{+}{1s^2}
    \lewis{F}{..}{..}{..}{..}{-}{1s^22s^22p^6}
    $\quad$(= LiF)
\end{center}
% \vspace{-0.6cm}
