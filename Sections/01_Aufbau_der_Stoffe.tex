\section{Aufbau der Stoffe}

\subsection{Grundlagen}
    \vspace{-0.6cm}
    \begin{align*}
        &\text{Atomare Masseeinheit:} &&u = \frac{1}{6.022\cdot10^{23}}g\\
        &\text{Elementarladung:} &&e = \pm 1.6022 \cdot 10^{-19}C\\
        &\text{max Elektronen pro Energieniveau:} &&Elektronen = 2 \cdot n^{2}
    \end{align*}
    \vspace{-0.1cm}
    Atome sind aufgebaut aus Protonen und Neutronen im Kern sowie Elektronen in der Hülle.
    %TODO: Schreibweise atomar

\subsection{Valenzelektronen}
    Die Anzahl Valenzelektronen (V.e.) kann anhand der Hauptgruppen aus dem PSE ausgelesen werden:
    \begin{itemize}
        \item Natrium(Na); erste Hauptgruppe = 1 V.e.
        \item Kohlenstoff(C); 4. Hauptgruppe = 4 V.e.
    \end{itemize}  
    Für die Elemente der Nebengruppe wird die bestimmung der V.e.-Anzahl komplizierter/unmöglich da diese in verschiedenen Formen vorkommen können.

    Die chemischen Eigenschaften der Elemente sind stark abhängig von der Anzahl-V.e. \\
    \textbf{Atome streben die Oktett-Regel an!} \\
    Das bedeutet das die Atome steht die äusserste Schale (Valenz-Schale) voll haben möchten. Um diesen Zustand zu erreichen werden Elektronen aufgenommen oder abgegeben(chemische Reaktion).

\subsection{Lewis-Formel $\rightarrow$ gibt nur Valenzelektronen an}
    % In dieser Schreibweise werden nur die Valenzelektronen der Elemente angegeben. Dies hilft dabei die entstehenden Bindungen zu visualisieren.
    \begin{center}
        \lewis{Li}{}{}{}{.}{}{} +
        \lewis{F}{..}{.}{..}{..}{}{}
        $\longrightarrow$
        \lewis{Li}{}{}{}{}{+}{}
        \lewis{F}{..}{..}{..}{..}{-}{}
        $\quad$(= LiF)
    \end{center}
    \vspace{-0.6cm}
