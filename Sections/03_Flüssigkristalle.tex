\section{Flüssigkristalle}
Flüssigkristalle haben zwischen den Aggregatzuständen ``fest'' und ``flüssig'' einen weiteren Aggregatzustand. Der ``flüssigkristalline'' Aggregatzustand macht sich erkennbar durch die trübe Farbe.        

Es wird in 3 verschiedene flüssigkristalline-Phasen unterschieden:
\begin{itemize}
    \item smektische Phase
    \item nematische Phase
    \item cholesterische Phase
\end{itemize}
Damit Moleküle eine solche Phase zeigen können müssen folgende Kriterien erfüllt sein:
\begin{itemize}
    \item lange, stäbchenartige Moleküle (4x - 6x Molekülbreite)
    \item starre Atomgruppen wie z.B. Benzen-Ringe, Doppel- Dreifachbindungen
    \item Funktionellegruppe mit sehr starken Dipolmoment (-CN-, -COOH)
\end{itemize}
\vspace{-0.2cm}
\begin{center}
    \includegraphics[height=1cm]{pictures/Flüss.png}
\end{center}
\vspace{-0.5cm}

\subsection{TN-Zelle}
\begin{minipage}[t]{0.5\linewidth}
    \includegraphics[width=\linewidth]{pictures/TN-Zelle1.png}
    ohne angelegte Spannung
\end{minipage}
\begin{minipage}[t]{0.5\linewidth}
    \includegraphics[width=\linewidth]{pictures/TN-Zelle2.png} 
    mit angelegter Spannung 
\end{minipage}

