\section{Redox-Reaktionen}

\subsection{Grundlagen}
    \begin{minipage}{0.5\linewidth}
        Reduktionsmittel (RM) = $e^{-}$-Spender

        Oxid\tikz[baseline=(text.base)]\node[fill=orange, fill opacity=0.3, text opacity=1, rounded corners, inner sep=2pt, minimum height=5pt] (text) {a};tion = $e^{-}$-Abg\tikz[baseline=(text.base)]\node[fill=orange, fill opacity=0.3, text opacity=1, rounded corners, inner sep=2pt, minimum height=5pt] (text) {a};be

        = Erhöhung der OZ
    \end{minipage}
    \hfill
    \begin{minipage}{0.5\linewidth}
        Oxidationsmittel (OM) = $e^{-}$-Akzeptor

        Red\tikz[baseline=(text.base)]\node[fill=green, fill opacity=0.3, text opacity=1, rounded corners, inner sep=2pt, minimum height=5pt] (text) {u};ktion = $e^{-}$-A\tikz[baseline=(text.base)]\node[fill=green, fill opacity=0.3, text opacity=1, rounded corners, inner sep=2pt, minimum height=5pt] (text) {u};fnahme

        = Erniedrigung der OZ
    \end{minipage}

    % Eine Reaktion ist eine Redox-Reaktion, wenn die Oxidationszahlen der Atome der Edukte nicht die selben sind wie die Oxidationszahlen der Atome der Produkte.
    
    \begin{minipage}{0.5\linewidth}
        \includegraphics[height=4cm]{pictures/Galv.png}
    \end{minipage}
    \hfill
    \begin{minipage}{0.5\linewidth}
        Beispiel: Zu gelöstem \ce{NaCl} und \ce{Pb(NO3)2} wird eine Eisenplatte hineingehalten:
        \begin{itemize}
            \item \ce{Na+} zu \ce{Fe} bergauf, Keine freiwillige Reaktion
            \item \ce{Pb^{2+}} zu \ce{Fe} bergab, freiwillige Reaktion
            \begin{itemize}
                \item Oxidation : \ce{Fe -> Fe^{2+} + 2e-}
                \item Reduktion : \ce{Pb^{2+} + 2e- -> Pb}
            \end{itemize}
        \end{itemize}
    \end{minipage}
\subsection{Redoxzahl Bestimmen}
Es gibt folgende Regeln für das Bestimmen der Oxidationszahlen:

\begin{tabular}{cll}
    \textbf{R1} & OZ elementarer Stoffe ist Null & \ce{O_2^0, Cl_2^0} \\
    \textbf{R2} & OZ einatomiger Ionen = Ionenladung & \makecell{\ce{Na^+} : \ce{Fe^{2+}} : +\Romannum{2}\\ \ce{Fe^{2+}} : +\Romannum{2}} \\
    \textbf{R3} & Bei Molekülen: Bindungselektron beim negativerem Atom & \makecell{\ce{F}: immer -\Romannum{1}\\ \ce{O}: meist -\Romannum{2} \\ \ce{H}: Meist +\Romannum{1}} \\
    \textbf{R4} & Summe aller OZ = Ladung des Teilchens & \\
    \textbf{R5} & Atome aus der ersten Hauptgruppe sind meist auch +\Romannum{1} & (\ce{Li, Na, K, ...})\\
\end{tabular}

    % Aufgrund der \textbf{Standartpotenziale} der Metalle Zn und Cu herrscht eine ``Spannun'', welche die Reaktion ermöglicht. 
    % Das \textbf{\ce{Zn^0}} wird an der Anode zu \textbf{\ce{Zn^2+}} oxidiert (\ce{e^-}-Abgabe), \textbf{\ce{Zn^0}} dient somit als Reduktionsmittel. 
    % Die Elektronen werden an die Kathode abgegeben, wo \textbf{\ce{Cu^2+}} aus der Lösung zu \textbf{\ce{Cu^0}} reduziert (\ce{e^-}-Aufnahme) wird. 
    % \textbf{\ce{Cu^2+}} dient somit als Oxidationsmittel. Damit die Lösungen jeweils ungeladen bleiben, wandern über die Salzbrücke 
    % \textbf{\ce{Zn^2+}-Ionen} und \textbf{\ce{SO4^2-}-Ionen}.
\subsection{Redoxpotential}
    Redoxpotential $\to$ auslesen aus Redoxreihe ganz rechts 
    
    Gemessen gegenüber Wasserstoff-Elektrode

    Abhängig von: pH, Druck, Ionenkonzentration, Temperatur $\to$ \textbf{Nernst-Gleichung:}

    % Das Redoxpotential einer Halbzelle kann aus der Redox-Reihe ausgelesen werden (ganz rechts). Dieses Potenzial wurde jeweils 
    % gegenüber einer Standart-Wasserstoff-Elektrode gemessen.
    
    % Das Redoxpotential ist jedoch von pH, Druck, Ionenkonz und Temperatur abhängig. Potenziale bei Nicht-Standardbedingungen können 
    % mit folgender GLeichung berechnet werden.


    \begin{minipage}{0.4\linewidth}
        \(
        \boxed{\bm{E^{0}_{\mathrm{RM/OM}} + \frac{0.059}{z} \cdot \lg \frac{[\mathrm{OM}]}{[\mathrm{RM}]}}}
        \)
    \end{minipage}
    \hfill
    \begin{minipage}{0.55\linewidth}
        \begin{itemize}
            \item z = Anz. \ce{e^-} die pro Atom übergeben werden
            \item \ce{[OM]} = konz. OM in mol/L
            \item \ce{[RM]} = konz. RM in mol/L, bei Metallen 1
        \end{itemize}
    \end{minipage}


     \begin{minipage}{0.65\linewidth}
        Inkl. pH-Wert:

        \includegraphics[height=2.8cm]{pictures/Nernstph.png}
    \end{minipage}
    \hfill
    \begin{minipage}{0.3\linewidth}
        Edle Metalle E > 0V
        \begin{itemize}
            \item Gold, Silber
            \item Platin, Palladium
        \end{itemize}
        Unedle Metalle E < 0V
        \begin{itemize}
            \item Natrium, Lithium
            \item Eisen, Zink
        \end{itemize}
    \end{minipage}