% ================================================================================================
\documentclass[8pt, a4paper, twoside]{extarticle}
%\documentclass[fontsize=8pt, a4paper, fleqn, landscape, DIV=calc]{scrartcl}
% Font size:    8pt
% Paper size:   A4
% style:        twoside (needed, so odd and even pages have different margins)

% ========================================= DOCUMENT INFO =========================================
\def\title{Elektrochemie}           					                            % title
\def\shorttitle{ElChem OST} 								                        % short title (displayed as PDF title)
\def\dozent{Mario Graf}         		                                            % lecturer
\def\semester{FS 2025} 									                            % semester
\def\author{Fabian Suter \& Steiner,  \\ Vorlage: Yves Looser, Nino Briker, Sandro Heidrich}  	% author(s)
\def\repo{https://github.com/FabianSuter/ElChem.git}                                % repository link
\def\version{0.0.1}   										                        % version
\def\pagelimit{10}           								                        % page limit -> causes pages after limit to be red
\def\titleoption{normal}   									                        % options: compact, normal
\def\enableToC{true}

% ================================= PACKAGES, SETUP AND COMMANDS ==================================
\input{preamble.tex}


% =========================================== DOCUMENT ============================================

\begin{document}
    \begin{layout}
        
        \section{Aufbau der Stoffe}

\subsection{Grundlagen}
\begin{center}
    \begin{tabular}{|c|c|c|}\hline
        Atomare Masseeinheit:               & Elementarladung:         & max. $2 \cdot n^{2}$ Elektronen \\ 
        $u = \frac{1}{6.022}10^{-23}g$      & $e = \pm 1.6022 \cdot 10^{-19}C$ & pro Energieniveau $n$ \\ \hline
    \end{tabular}
\end{center}


\begin{minipage}{0.25\linewidth}
    $ \prescript{\text{Atommasse }}{\text{Ordnungszahl }}{X}^{\text{ Ladung}} $
\end{minipage}
\hfill
\begin{minipage}{0.65\linewidth}
    \begin{itemize}[itemsep=1pt, parsep=0pt]
        \item Ordnungszahl = Protonenzahl
        \item Atommasse = Summe Protonen \& Neutronen
        \item Ladung = Summe Protonen \& Elektronen
    \end{itemize}
\end{minipage}

Bausteine: Protonen \& Neutronen im Kern, Elektronen in der Hülle

\subsection{Valenzelektronen (Ve) ...}

\begin{itemize}[itemsep=1pt, parsep=0pt]
    \item -Anzahl anhand der \textbf{Hauptgruppen} im PSE bestimmbar
    \begin{itemize}
        \item Natrium (Na): 1. Hauptgruppe = 1 Ve
        \item Kohlenstoff (C): 4. Hauptgruppe = 4 Ve 
    \end{itemize}
    \item -Bestimmung der \textbf{Nebengruppen} komplizierter/unmöglich 
    
        $\Rightarrow$ verschiedene Formen möglich

    \item haben starken Einfluss auf chemische Eigenschaften
    \item variieren zw. 1 und 8 $\rightarrow$ \textbf{Oktett-Regel}
    
        $\rightarrow$ Atome wollen äusserste Schale voll haben
        
        $\rightarrow$ Elektronen abgeben oder aufnehmen $\Rightarrow$ chemische Reaktion
\end{itemize}

% Anzahl Valenzelektronen (Ve) kann anh. der \textbf{Hauptgruppen} im PSE bestimmt werden:
% \begin{itemize}
%     \item Natrium (Na): 1. Hauptgruppe = 1 Ve
%     \item Kohlenstoff (C): 4. Hauptgruppe = 4 Ve
% \end{itemize}
% Für die Elemente der \textbf{Nebengruppe} ist die Bestimmung der Ve-Anzahl \textbf{komplizierter/unmöglich} $\rightarrow$ verschiedene Formen möglich

% Die chemischen Eigenschaften der Elemente sind stark abhängig von der Anzahl-Ve 

% \textbf{Atome streben die Oktett-Regel an!}

% Das bedeutet das die Atome steht die äusserste Schale (Valenz-Schale) voll haben möchten. 
% Um diesen Zustand zu erreichen werden Elektronen aufgenommen oder abgegeben(chemische Reaktion).

\subsection{Lewis-Formel $\rightarrow$ gibt nur Valenzelektronen an}
\begin{center}
    \lewis{Li}{}{}{}{.}{}{1s^22s^1} +
    \lewis{F}{..}{.}{..}{..}{}{1s^22s^22p^5}
    $\longrightarrow$
    \lewis{Li}{}{}{}{}{+}{1s^2}
    \lewis{F}{..}{..}{..}{..}{-}{1s^22s^22p^6}
    $\quad$(= LiF)
\end{center}
% \vspace{-0.6cm}

        \section{Stoffklassen}
    Stoffe lassen sich in 3 Arten einteilen:
    \begin{itemize}
        \item molekulare Stoffe:\\
        Abgeschlossener Atomverband aus Nichtmetallen (Molekül).\\Formel: genaue Anzahl Atome pro Molekül. Z.B \ce\\Nicht elektrisch Leitend, da keine freien Ladungsträger vorhanden.
        \item Metalle und Halbmetalle:\\
        unendlicher Verband aus metallischen Atomkernen umgebend von delokaliserten (Valenz-) Elektronen (Elektronen-Wolke).\\Formel: Verhältnis der Atome im Gitter. Z.B. \ce{Fe}
        \item Salze:\\
        unendlicher Verband aus Ionen(Kation(\ce{+}); Anion(\ce{-})).\\ metallische Kationen und nichtmetallische Kationen\\(können auch molekulare Kationen sein(\ce{SO4-})).\\Formel: Verhältnis der Kationen und Anionen. Z.B. \ce{KCl}\\Besitzt in Schmelze und in Lösung frei Ladungsträger (Ionen) leiten in diesen Zuständen dementsprechend gut Strom.
    \end{itemize}
\subsection{Metalle und Halbmetalle}
    Metalle besitzen durch die delokalisierten Valenzelektronen (Elektronenwolke) frei Ladungsträger, leiten sehr gut Strom und Wärme.
    \begin{itemize}
        \item Leitfähigkeit nimmt mit steigender Temperatur ab.\\Die Bewegung der Atomrümpfe erhöht sich wodurch weniger Platz für die ELektronen um sich zu bewegen bleibt.
    \end{itemize}
    \begin{center}
    \includegraphics[height=3cm]{pictures/BBànder.png}
    \end{center}
    Allgemein sind Stoffe leitfähig wenn sie entweder wie Lithium:
    \begin{itemize}
        \item das Valenzband (spez. Energieniveau) nicht ganz gefüllt haben und sich dadurch ELektronen in jenem Band bewegen können.
    \end{itemize}
    oder wenn sie wie Beryllium:
    \begin{itemize}
        \item das Valenzband komplett gefüllt haben dieses jedoch mit einem leeren Leitungsband überlappt. Wodurch wiederum die beweglichkeit der Elektronen gewährleistet ist.
    \end{itemize}
    Halbmetalle haben weder Elektronenwolken noch überlappende Energieniveaus jedoch sind Valenz- und Leitungsband so nahe bei einander das ein überspringen ermöglicht wird.
    \begin{itemize}
        \item Leitfähigkeit nimmt mit zunehmender Temperatur stark zu.\\Die Elektronen springen viel zahlreicher auf das Leitungband über wodurch im Leitungsband wiederum Platz für Elektronenbewegung geschaffen wird.
    \end{itemize}
\subsection{Dotierung von Halbmetallen}
    Unter Dotierung versteht man das einbringen von Fremdatomen ins Atomgitter eines Halbleiters. Man unterscheidet 2 Arten von dotierung:
    \begin{itemize}
        \item n-Halbleiter\\
        z.B. einzelne As-Atome im Si-Gitter(1:10'000'000)\\Ein ``überschüssiges'' Elektron pro As-Atom. Dadurch entsteht Leitfähigkeit. Elektron von As-Atom kann ins Leitungsband von Si überspringen und sich dort frei bewegen.
        \item p-Halbleiter
        z. B. einzelne B-Atome im Si-Gitter(1:1'000'000)\\Ein ``fehlendes'' Elektron pro B-Atom. Dadurch entsteht Leitfähigkeit. Elektronen aus dem vollen Valenzband von Si können in diese "Lücke" springen und sich so bewegen.
    \end{itemize}
\subsection{Bindungswinkel}
    \begin{center}
    \includegraphics[height=4cm]{pictures/Winkel.png}
    \end{center}
\subsection{Löslichkeit}
    Die Löslichkeit von Salzen hängt von ihrer Bildungsstärke ab. Je grösser die Ladung der Ionen und je grösser die Ionen desto schlechter sind sie in Wasser löslich.
    \begin{center}
    \includegraphics[height=2cm]{pictures/Löslichkeit.png}
    \end{center}
        \section{Flüssigkristalle}
Flüssigkristalle haben zwischen den Aggregatzuständen ``fest'' und ``flüssig'' einen weiteren Aggregatzustand. Der ``flüssigkristalline'' Aggregatzustand macht sich erkennbar durch die trübe Farbe.        

Es wird in 3 verschiedene flüssigkristalline-Phasen unterschieden:
\begin{itemize}
    \item smektische Phase
    \item nematische Phase
    \item cholesterische Phase
\end{itemize}
Damit Moleküle eine solche Phase zeigen können müssen folgende Kriterien erfüllt sein:
\begin{itemize}
    \item lange, stäbchenartige Moleküle (4x - 6x Molekülbreite)
    \item starre Atomgruppen wie z.B. Benzen-Ringe, Doppel- Dreifachbindungen
    \item Funktionellegruppe mit sehr starken Dipolmoment (-CN-, -COOH)
\end{itemize}
\vspace{-0.2cm}
\begin{center}
    \includegraphics[height=1cm]{pictures/Flüss.png}
\end{center}
\vspace{-0.5cm}

\subsection{TN-Zelle}
\begin{minipage}[t]{0.5\linewidth}
    \includegraphics[width=\linewidth]{pictures/TN-Zelle1.png}
    ohne angelegte Spannung
\end{minipage}
\begin{minipage}[t]{0.5\linewidth}
    \includegraphics[width=\linewidth]{pictures/TN-Zelle2.png} 
    mit angelegter Spannung 
\end{minipage}


        \section{Ablauf chemischer Reaktionen (Freiwilligkeit)}

\subsection{Enthalpie H / Reaktionsenthalpie (Wärme) $\Delta H_R$}
    Prinzip Energieminimum: Stoff will energiearmen Zustand erreichen!

    $\Delta H_R = H_{Produkte} - H_{Edukte} \quad [H] = \frac{\kilo\joule}{\mol \cdot \kelvin}$

    $\Delta H_R < 0 \Rightarrow$ \textcolor{red}{exotherm} $\qquad \Delta H_R > 0 \Rightarrow$ \textcolor{blue}{endotherm}

\subsection{Entropie S (Unordnung) / Reaktionsentropie $\Delta S_R$}
    Prinzip Energiemax: alle Stoffe und Systeme wollen möglichst grosse Entropie

    \begin{tabular}{p{5.5cm}l}
        $\Delta S_R = \sum S^0_{Produkte} - \sum S^0_{Edukte}$ & $[\Delta S_R] = \frac{\color{red}\joule}{\mol \cdot \kelvin}$
    \end{tabular}\\
    $S^0$: Molare Standartentropie (1mol des Stoffs bei Std.Bedingungen)
\subsection{Freie Enthalpie $\Delta G$}
    Beschreibt Freiwilligkeit der Reaktion:

    \begin{tabular}{p{5.5cm}l}
        $\Delta G = \Delta H - T \cdot \Delta S$ & $[\Delta G] = \frac{\kilo\joule}{\mol}$
    \end{tabular}
    \begin{itemize}
        \item $\Delta G < 0:$ Exergon (freiwillige Reaktion)
        \item $\Delta G > 0:$ Endergon(unfreiwillige Reaktion)
    \end{itemize}

\subsection{Aktivierungsenergie/Reaktionsgeschw./Katalysatoren}
\begin{minipage}[t]{0.3\linewidth}
    \vspace*{0pt}
    \includegraphics[width=\linewidth]{pictures/Katalysator.png}
\end{minipage}
\begin{minipage}[t]{0.7\linewidth}
    \begin{itemize}
        \item RGT-Regel: $\Delta$ T = 10 $\rightarrow$ RG $\cdot$ 2
        \item Katalysator = Stoff nimmt an Reaktion teil, wird nicht verbraucht 
        \item Beschleunigt Reaktion: $E_{AKat} \ll E_{ANorm}$
        \item $\Delta$ G sowie $\Delta H_R$ bleiben gleich
        \item Selektiv (wirkt nicht mit allen Stoffen)
    \end{itemize}
\end{minipage}

        \section{Säure-Base-Reaktionen}
$pK_{s}+ pK_{b} = 14$
\subsection{Säure-Base GGW}
Bergab = GGW rechts: \ce{ HCl + H2O <=>> Cl- + H3O+ }\\
Bergauf = GGw links: \ce{ HS- + H2O <<=> S2- + H3O+ }\\




        \section{Redox-Reaktionen}

\subsection{Grundlagen}
Eine Reaktion ist eine Redox-Reaktion wenn die Oxidationszahlen der Atome der Edukte nicht die selben sind wie die Oxidationszahlen der Atome der Produkte.
\begin{center}
    \includegraphics[height=4cm]{pictures/Galv.png}
\end{center}
Aufgrund der \textbf{Stadnartpotenziale} der Metalle Zn und Cu herrscht eine "Spannung" welche die Reaktion ermöglicht. Das \textbf{\ce{Zn^0}} wird an der Anode zu \textbf{\ce{Zn^2+}} oxidiert (\ce{e^-}-Abgabe), \textbf{\ce{Zn^0}} dient somit als Reduktionsmittel. Die Elektronen werden an die Kathode abgegeben wo \textbf{\ce{Cu^2+}} aus der Lösung zu \textbf{\ce{Cu^0}} reduziert (\ce{e^-}-Aufnahme) wird. \textbf{\ce{Cu^2+}} dient somit als Oxidationsmittel. Damit die Lösungen jeweils ungeladen bleiben wandern über die Salzbrücke \textbf{\ce{Zn^2+}-Ionen} und \textbf{\ce{SO4^2-}-Ionen}.
\subsection{Redoxpotential}
Das Redoxpotential einer Halbzelle kann aus der Redox-Reihe ausgelesen werden (ganz rechts). Dieses Potenzial wurde jeweils gegenüber einer Standart-Wasserstoff-Elektrode gemessen.\\ Das Redoxpotential ist jedoch von pH, Druck, Ionenkonz und Temperatur abhängig. Potenziale bei nichtstandart bedingungen können mit folgender GLeichung berechnet werden.\\
Nernst-Gleichung:\\
\begin{center}
    \includegraphics[height=1cm]{pictures/Nernst.png}
\end{center}
z = Anz. \ce{e^-} die pro Atom übergeben werden.\\\ce{[OM]} = konz. OM in mol/L\\\ce{[RM]} = konz. RM in mol/L\\
Sehr ähnlich wird die Berechnung wenn der pH-Wert berücksichtig wird:
\begin{center}
    \includegraphics[height=3cm]{pictures/Nernstph.png}
\end{center}
        \section{Anwendungen der Redox-Reaktionen}
   Spannung galvanische Zelle: $U = \ce{E^{Kathode}} - \ce{E^{Anode}} V$
% TODO: mehr einfügen, ansonsten killen

        \section{Korrosion}
Metall reagiert als RM: \ce{Me <--> $Me^{z+}$ + z e-}, Möglich wenn $\Delta$G $<$ 0 $\&$ v.a. \ce{O2, H2O/H3O+} (OM) vorhanden.
\subsection{Korrosionsarten}
\subsubsection{Elchem Korrosion}
$\Rightarrow$ Bildung galvanische Zelle
\subsubsection{O2-Typ-Korrosion}
\ce{O2 + 2H2O + 4e- -> 4OH-}\\
\ce{2Fe + 3/2 O2 + H2O -> 2FeOOH}\\
Voraussetzung ist vorhandensein von \ce{O2} und \ce{H2O}; RG relativ langsam!
\subsubsection{Säure/Wasserstoffkorrosion}
Ist pH-Abhängig:
- Sauer: \ce{2H3O+ + 2 e- <--> H2 + 2H20}\\
- Basisch: \ce{2H2O + 2 e- <--> H2 + 2 OH-}\\
\subsubsection{Beispiele Al H-Typ-korrosion}
\textbf{H2 Korrosion von Al in basischer Lösung}\\
\begin{tabular}{p{1.8cm}c}
Depassivierung: & \ce{Al2+2OH-+3H2O -> 2[AL(OH)4]-(aq)}\\
Oxidation:      & \ce{Al -> Al3+ + 3e-}\\
Reduktion:      & \ce{2H2O + 2 e- -> H2 + 2OH-}\\
Redoxreaktion:  & \ce{2 Al + 6 H2O + 2OH- -> 2[Al(OH)4]- + 3H2}
\end{tabular}
\textbf{H2 Korrosion von Al in saurer Lösung}\\
\begin{tabular}{p{1.8cm}c}
Oxidation:      & \ce{Al -> Al3+ + 3e-}\\
Reduktion:      & \ce{2H3O+ + 2e- -> H2 + H2 + 2 H2O}\\
Redoxreaktion:  & \ce{2 Al + 6 H3O+ -> 2 Al3+ + 6H2O + 3H2}
\end{tabular}
\subsection{Oxidschichten}
Metallische Werkstoffe (Ausser Gold/Platinmetalle) bilden bei Raumtemperatur mit Luft eine Oxidschicht, es entsteht ein Metalloxid:\\
\ce{n Me + \frac{m}{2} O2 -> $Me_nO_m$}\\
Der schutzfaktor kann mittels PBV ermittelt werden:\\
$PBV = \frac{V(Metalloxid)}{V(Metall)}$\\
\begin{itemize}
    \item PBV $\ll$ 1: Rissige, nicht schützende Schicht\\
    bsp. MG(PBV = 0.8), Na (0.3)
    \item PBV 1- ca.2: Kompakte, schützende Oxidschicht\\
    bsp. Al(1.3), Ni(1.5), Ti(1.7), Cu(1.7), Cr(2.1), Fe(2.1)
    \item PBV $\gg$ 2: Abbläternde nicht schützende Schicht\\
    bsp. V(3.2), W(3.4), Rost(3.6)
\end{itemize}
\subsection{Ablauf der Korrosion in wässrigen Lösungen}
Alle Korrosionsreaktionen verlaufen in 2 Teilschritten:\\
\begin{itemize}
    \item Depassivierung
    \item Eigentliche Korrosion
\end{itemize}
Voraussetzungen für Korrosion:\\
\begin{itemize}
    \item Metall ist in Elektrolytlösung eingetaucht
    \item Metall ist von dünnem Flüssigkeitsfilm bedekt.\\
    Können durch Regen, Tau, Bodenfeuchtigkeit oder rel. Luftfeuchtigkeit $> 70\%$ entstehen. Bei Oberflächen mit hygroskopischen Salzen kann auch früher Korrosion entstehen.
\end{itemize}
\subsection{Passivatoren und Depassivatoren}
Ob Depassivierung eines met. werkstoffes möglich ist ist vom Gehalt von Passivatoren und Depassivatoren in Elektrolytlösung.
\subsubsection{Passivatoren}
$\Rightarrow$ bieten \textbf{anodischen Schutz}($E_A$ wird vergrössert)\\
Passivatoren für Fe: \ce{OH-, CrO4^{2-}, NO2-}\\
Passivator für Al: \ce{NO3-}
\subsubsection{Depasivatoren}
$\Rightarrow$ \textbf{zerstören Passivoxidfilm}, bewirken (oft lokale \textbf{Depassivierung}($E_A$ wird verkleinert)\\
Depassivatoren für Fe: \color{blue} \ce{Cl-}\color{black}, \color{red} \ce{H3O+}\color{black}, \ce{SO4^2-}\\
Depassivatoren für Al: \color{blue} \ce{Cl-}\color{black}, \color{red} \ce{H3O+}\color{black}, \ce{OH-}\\
Depassivatoren für Cu: \color{blue} \ce{Cl-}\color{black}, \color{red} \ce{H3O+}\color{black}, \ce{NH3}\\
Depassivatoren für Ni: \color{blue} \ce{Cl-}\color{black}, \color{red} \ce{H3O+}\color{black}\\
\subsection{Potentialverhältnisse/Aktivierungsenergie}
Wann korrodieren Metalle nach \ce{H2}/\ce{O2}-Typ?\\
$\Rightarrow \Delta$G $<$ 0\\
$\Rightarrow$ Korrosion abhängig von E(M/\ce{M^z+}) unsd E(OM)\\
E(OM) ist pH-abhängig:\\
$E_{H2}$ = -0.059*pH\\
$E_{O2}$ = 1.23 - 0.059*pH\\
\subsection{Kontaktkorrosion}
Reduktion von \ce{O2} an gesamter Oberfläche
Oxidation nur an unedlerem Metall $\rightarrow$ verstärkte Korrosion\\
Edleres Metall $\rightarrow$ keine Korrosion(kathodisch geschützt)\\
Flächenregel: $\frac{v_k(Zn)}{v_k(Zn + Fe)} = \frac{A(Zn)}{A(Zn + Fe)}$
\subsection{Lochfrasskorrosion}
Stark lokalisierte Korrosion\\
Bildung enger tiefer Löcher\\
schwer erkennbar\\
\subsection{Belüftungselemente}
Kann nur bei passivierbaren Metallen auftreten!\\
Für Passivschicht ist \ce{O2} notwendig\\
An engen stellen kann \ce{O2} zufuhr erschwert werden. Dies bewirkt lokale Depassivierung $\rightarrow$ lochfrass\\
Zusätzlich Flächenregel(Spalt = kleine Anode, Passivoxidschicht = grosse Kathode

        
        % \section{Emotional support meme}
        % \begin{center}
        %     \includegraphics[width=\columnwidth]
        %     {pictures/argon.jpeg} 
        % \end{center}

    \end{layout}	
\end{document}
